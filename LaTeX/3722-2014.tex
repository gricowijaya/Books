%% ================================================================================
%% This LaTeX file was created by AbiWord.                                         
%% AbiWord is a free, Open Source word processor.                                  
%% More information about AbiWord is available at http://www.abisource.com/        
%% ================================================================================

\documentclass[a4paper,portrait,12pt]{article}
\usepackage[latin1]{inputenc}
\usepackage{calc}
\usepackage{setspace}
\usepackage{fixltx2e}
\usepackage{graphicx}
\usepackage{multicol}
\usepackage[normalem]{ulem}
%% Please revise the following command, if your babel
%% package does not support en-US
\usepackage[en]{babel}
\usepackage{color}
\usepackage{hyperref}
 
\begin{document}


\begin{flushleft}
LATEX for Beginners
\end{flushleft}


\begin{flushleft}
Workbook
\end{flushleft}


\begin{flushleft}
Edition 5, March 2014
\end{flushleft}


\begin{flushleft}
Document Reference: 3722-2014
\end{flushleft}





\begin{flushleft}
\newpage
\newpage
Preface
\end{flushleft}


\begin{flushleft}
This is an absolute beginners guide to writing documents in LATEX using
\end{flushleft}


\begin{flushleft}
TeXworks. It assumes no prior knowledge of LATEX, or any other computing
\end{flushleft}


\begin{flushleft}
language.
\end{flushleft}


\begin{flushleft}
This workbook is designed to be used at the {`}LATEX for Beginners' student
\end{flushleft}


\begin{flushleft}
iSkills seminar, and also for self-paced study. Its aim is to introduce an
\end{flushleft}


\begin{flushleft}
absolute beginner to LATEX and teach the basic commands, so that they can
\end{flushleft}


\begin{flushleft}
create a simple document and find out whether LATEX will be useful to them.
\end{flushleft}





\begin{flushleft}
If you require this document in an alternative format, such as large print,
\end{flushleft}


\begin{flushleft}
please email is.skills@ed.ac.uk.
\end{flushleft}


\begin{flushleft}
Copyright c IS 2014
\end{flushleft}


\begin{flushleft}
Permission is granted to any individual or institution to use, copy or redistribute this document whole or in part, so long as it is not sold for profit and
\end{flushleft}


\begin{flushleft}
provided that the above copyright notice and this permission notice appear
\end{flushleft}


\begin{flushleft}
in all copies.
\end{flushleft}


\begin{flushleft}
Where any part of this document is included in another document, due acknowledgement is required.
\end{flushleft}





\begin{flushleft}
i
\end{flushleft}





\begin{flushleft}
\newpage
ii
\end{flushleft}





\begin{flushleft}
\newpage
Contents
\end{flushleft}


\begin{flushleft}
1 Introduction
\end{flushleft}


\begin{flushleft}
1.1 What is LATEX? . . . . . . . . . . . . . . . . . . . . . . . . . .
\end{flushleft}


\begin{flushleft}
1.2 Before You Start . . . . . . . . . . . . . . . . . . . . . . . . .
\end{flushleft}





1


1


2





\begin{flushleft}
2 Document Structure
\end{flushleft}


\begin{flushleft}
2.1 Essentials . . . . .
\end{flushleft}


\begin{flushleft}
2.2 Troubleshooting . .
\end{flushleft}


\begin{flushleft}
2.3 Creating a Title . .
\end{flushleft}


\begin{flushleft}
2.4 Sections . . . . . .
\end{flushleft}


\begin{flushleft}
2.5 Labelling . . . . . .
\end{flushleft}


\begin{flushleft}
2.6 Table of Contents .
\end{flushleft}





.


.


.


.


.


.





3


3


5


5


6


7


8





.


.


.


.


.


.





11


11


11


12


13


14


15





.


.


.


.


.


.





\begin{flushleft}
3 Typesetting Text
\end{flushleft}


\begin{flushleft}
3.1 Font Effects . . . . .
\end{flushleft}


\begin{flushleft}
3.2 Coloured Text . . . .
\end{flushleft}


\begin{flushleft}
3.3 Font Sizes . . . . . .
\end{flushleft}


\begin{flushleft}
3.4 Lists . . . . . . . . .
\end{flushleft}


\begin{flushleft}
3.5 Comments \& Spacing
\end{flushleft}


\begin{flushleft}
3.6 Special Characters .
\end{flushleft}





.


.


.


.


.


.





.


.


.


.


.


.





.


.


.


.


.


.





.


.


.


.


.


.





.


.


.


.


.


.





.


.


.


.


.


.





.


.


.


.


.


.





.


.


.


.


.


.





.


.


.


.


.


.





.


.


.


.


.


.





.


.


.


.


.


.





.


.


.


.


.


.





.


.


.


.


.


.





.


.


.


.


.


.





.


.


.


.


.


.





.


.


.


.


.


.





.


.


.


.


.


.





.


.


.


.


.


.





.


.


.


.


.


.





.


.


.


.


.


.





.


.


.


.


.


.





.


.


.


.


.


.





.


.


.


.


.


.





.


.


.


.


.


.





.


.


.


.


.


.





.


.


.


.


.


.





.


.


.


.


.


.





.


.


.


.


.


.





.


.


.


.


.


.





.


.


.


.


.


.





.


.


.


.


.


.





.


.


.


.


.


.





.


.


.


.


.


.





.


.


.


.


.


.





.


.


.


.


.


.





.


.


.


.


.


.





.


.


.


.


.


.





.


.


.


.


.


.





.


.


.


.


.


.





.


.


.


.


.


.





.


.


.


.


.


.





.


.


.


.


.


.





.


.


.


.


.


.





.


.


.


.


.


.





\begin{flushleft}
4 Tables
\end{flushleft}


17


\begin{flushleft}
4.1 Practical . . . . . . . . . . . . . . . . . . . . . . . . . . . . . . 18
\end{flushleft}


\begin{flushleft}
5 Figures
\end{flushleft}


21


\begin{flushleft}
5.1 Practical . . . . . . . . . . . . . . . . . . . . . . . . . . . . . . 22
\end{flushleft}


\begin{flushleft}
6 Equations
\end{flushleft}


23


\begin{flushleft}
6.1 Inserting Equations . . . . . . . . . . . . . . . . . . . . . . . . 23
\end{flushleft}


\begin{flushleft}
6.2 Mathematical Symbols . . . . . . . . . . . . . . . . . . . . . . 24
\end{flushleft}


\begin{flushleft}
6.3 Practical . . . . . . . . . . . . . . . . . . . . . . . . . . . . . . 26
\end{flushleft}


\begin{flushleft}
7 Inserting References
\end{flushleft}





27


\begin{flushleft}
iii
\end{flushleft}





\newpage
7.1


7.2


7.3


7.4


7.5


7.6





\begin{flushleft}
Introduction . . . . . . . .
\end{flushleft}


\begin{flushleft}
The BibTeX file . . . . . .
\end{flushleft}


\begin{flushleft}
Inserting the bibliography
\end{flushleft}


\begin{flushleft}
Citing references . . . . .
\end{flushleft}


\begin{flushleft}
Styles . . . . . . . . . . .
\end{flushleft}


\begin{flushleft}
Practical . . . . . . . . . .
\end{flushleft}





.


.


.


.


.


.





.


.


.


.


.


.





\begin{flushleft}
8 Further Reading
\end{flushleft}





.


.


.


.


.


.





.


.


.


.


.


.





.


.


.


.


.


.





.


.


.


.


.


.





.


.


.


.


.


.





.


.


.


.


.


.





.


.


.


.


.


.





.


.


.


.


.


.





.


.


.


.


.


.





.


.


.


.


.


.





.


.


.


.


.


.





.


.


.


.


.


.





.


.


.


.


.


.





.


.


.


.


.


.





.


.


.


.


.


.





.


.


.


.


.


.





.


.


.


.


.


.





.


.


.


.


.


.





27


27


28


29


29


30


31





\begin{flushleft}
iv
\end{flushleft}





\begin{flushleft}
\newpage
Chapter 1
\end{flushleft}


\begin{flushleft}
Introduction
\end{flushleft}


1.1





\begin{flushleft}
What is LATEX?
\end{flushleft}





\begin{flushleft}
LATEX (pronounced lay-tek ) is a document preparation system for producing
\end{flushleft}


\begin{flushleft}
professional-looking documents, it is not a word processor. It is particularly
\end{flushleft}


\begin{flushleft}
suited to producing long, structured documents, and is very good at typesetting equations. It is available as free software for most operating systems.
\end{flushleft}


\begin{flushleft}
LATEX is based on TEX, a typesetting system designed by Donald Knuth in
\end{flushleft}


\begin{flushleft}
1978 for high quality digital typesetting. TEX is a low-level language that
\end{flushleft}


\begin{flushleft}
computers can work with, but most people would find difficult to use; so
\end{flushleft}


\begin{flushleft}
LATEX has been developed to make it easier. The current version of LATEX is
\end{flushleft}


\begin{flushleft}
LATEX2e.
\end{flushleft}


\begin{flushleft}
If you are used to producing documents with Microsoft Word, you will find
\end{flushleft}


\begin{flushleft}
that LATEX is a very different style of working. Microsoft Word is {`}What You
\end{flushleft}


\begin{flushleft}
See Is What You Get' (WYSIWYG), this means that you see how the final
\end{flushleft}


\begin{flushleft}
document will look as you are typing. When working in this way you will
\end{flushleft}


\begin{flushleft}
probably make changes to the document's appearance (such as line spacing,
\end{flushleft}


\begin{flushleft}
headings, page breaks) as you type. With LATEX you do not see how the final
\end{flushleft}


\begin{flushleft}
document will look while you are typing it --- this allows you to concentrate
\end{flushleft}


\begin{flushleft}
on the content rather than appearance.
\end{flushleft}


\begin{flushleft}
A LATEX document is a plain text file with a .tex file extension. It can be typed
\end{flushleft}


\begin{flushleft}
in a simple text editor such as Notepad, but most people find it is easier to
\end{flushleft}


\begin{flushleft}
use a dedicated LATEX editor. As you type you mark the document structure
\end{flushleft}


\begin{flushleft}
(title, chapters, subheadings, lists etc.) with tags. When the document
\end{flushleft}


\begin{flushleft}
is finished you compile it --- this means converting it into another format.
\end{flushleft}


\begin{flushleft}
Several different output formats are available, but probably the most useful
\end{flushleft}


1





\begin{flushleft}
\newpage
is Portable Document Format (PDF), which appears as it will be printed and
\end{flushleft}


\begin{flushleft}
can be transferred easily between computers.
\end{flushleft}





1.2





\begin{flushleft}
Before You Start
\end{flushleft}





\begin{flushleft}
The following conventions are used throughout this workbook:
\end{flushleft}


\begin{flushleft}
$\bullet$ Actions for you to carry out are bulleted with an arrow \`{o}.
\end{flushleft}


\begin{flushleft}
$\bullet$ Text you type is written in this font.
\end{flushleft}


\begin{flushleft}
$\bullet$ Menu commands and button names are shown in bold.
\end{flushleft}


\begin{flushleft}
Although the code in this workbook should work in any LATEX editor, specific
\end{flushleft}


\begin{flushleft}
examples and screenshots refer to TeXworks 0.4.5 on Windows 7.
\end{flushleft}





2





\begin{flushleft}
\newpage
Chapter 2
\end{flushleft}


\begin{flushleft}
Document Structure
\end{flushleft}


2.1





\begin{flushleft}
Essentials
\end{flushleft}





\begin{flushleft}
\`{o} Start TeXworks.
\end{flushleft}


\begin{flushleft}
A new document will automatically open.
\end{flushleft}


\begin{flushleft}
\`{o} Go to the Format menu and select Line Numbers.
\end{flushleft}


\begin{flushleft}
Line numbers are not essential, but will make it easier to compare your code
\end{flushleft}


\begin{flushleft}
with the screenshots and find errors.
\end{flushleft}


\begin{flushleft}
\`{o} Go to the Format menu and select Syntax Coloring, then LaTeX.
\end{flushleft}


\begin{flushleft}
Syntax colouring will highlight commands in blue and can make it easier to
\end{flushleft}


\begin{flushleft}
spot mistakes.
\end{flushleft}


\begin{flushleft}
\`{o} Type the following:
\end{flushleft}


\begin{flushleft}
\ensuremath{\backslash}documentclass[a4paper,12pt]\{article\}
\end{flushleft}


\begin{flushleft}
\ensuremath{\backslash}begin\{document\}
\end{flushleft}


\begin{flushleft}
A sentence of text.
\end{flushleft}


\begin{flushleft}
\ensuremath{\backslash}end\{document\}
\end{flushleft}


3





\begin{flushleft}
\newpage
The \ensuremath{\backslash}documentclass command must appear at the start of every LATEX
\end{flushleft}


\begin{flushleft}
document. The text in the curly brackets specifies the document class. The
\end{flushleft}


\begin{flushleft}
article document class is suitable for shorter documents such as journal
\end{flushleft}


\begin{flushleft}
articles and short reports. Other document classes include report (for longer
\end{flushleft}


\begin{flushleft}
documents with chapters, e.g. PhD theses), proc (conference proceedings),
\end{flushleft}


\begin{flushleft}
book and slides. The text in the square brackets specifies options --- in this
\end{flushleft}


\begin{flushleft}
case it sets the paper size to A4 and the main font size to 12pt.
\end{flushleft}


\begin{flushleft}
The \ensuremath{\backslash}begin\{document\} and \ensuremath{\backslash}end\{document\} commands enclose the text
\end{flushleft}


\begin{flushleft}
and commands that make up your document. Anything typed before \ensuremath{\backslash}begin
\end{flushleft}


\begin{flushleft}
\{document\} is known as the preamble, and will affect the whole document.
\end{flushleft}


\begin{flushleft}
Anything typed after \ensuremath{\backslash}end\{document\} is ignored.
\end{flushleft}


\begin{flushleft}
The empty lines aren't necessary1 , but they will make it easier to navigate
\end{flushleft}


\begin{flushleft}
between the different parts of the document as it gets longer.
\end{flushleft}





\begin{flushleft}
\`{o} Click on the Save button.
\end{flushleft}


\begin{flushleft}
\`{o} Create a new folder called LaTeX course in Libraries$>$Documents.
\end{flushleft}


\begin{flushleft}
\`{o} Name your document Doc1 and save it as a TeX document in this
\end{flushleft}


\begin{flushleft}
folder.
\end{flushleft}





\begin{flushleft}
It is a good idea to keep each of your LATEX documents in a separate folder
\end{flushleft}


\begin{flushleft}
as the compiling process creates multiple files.
\end{flushleft}





\begin{flushleft}
\`{o} Make sure the typeset menu is set to pdfLaTeX.
\end{flushleft}


\begin{flushleft}
\`{o} Click on the Typeset button.
\end{flushleft}


\begin{flushleft}
There will be a pause while your document is being converted to a PDF
\end{flushleft}


\begin{flushleft}
file. When the compiling is complete TeXworks' PDF viewer will open and
\end{flushleft}


\begin{flushleft}
display your document. The PDF file is automatically saved in the same
\end{flushleft}


\begin{flushleft}
folder as the .tex file.
\end{flushleft}


1





\begin{flushleft}
See section 3.5 on page 14 for information about how LATEX deals with empty space
\end{flushleft}


\begin{flushleft}
in the .tex file.
\end{flushleft}





4





\newpage
2.2





\begin{flushleft}
Troubleshooting
\end{flushleft}





\begin{flushleft}
If there is an error in your document and TeXworks cannot create the PDF
\end{flushleft}


\begin{flushleft}
the Typeset button will change to red with a white X (Abort typesetting
\end{flushleft}


\begin{flushleft}
button) and the Console output at the bottom of the screen will stay open.
\end{flushleft}


\begin{flushleft}
If this happens:
\end{flushleft}


\begin{flushleft}
\`{o} Click on the Abort typesetting button.
\end{flushleft}


\begin{flushleft}
\`{o} Read the Console output - the last line will probably include a line
\end{flushleft}


\begin{flushleft}
number and the command that caused the error.
\end{flushleft}


\begin{flushleft}
\`{o} Go to the line number in your document and fix the error.
\end{flushleft}


\begin{flushleft}
\`{o} Click on the Typeset button again.
\end{flushleft}





2.3





\begin{flushleft}
Creating a Title
\end{flushleft}





\begin{flushleft}
The \ensuremath{\backslash}maketitle command creates a title. You need to specify the title of
\end{flushleft}


\begin{flushleft}
the document. If the date is not specified today's date is used. Author is
\end{flushleft}


\begin{flushleft}
optional.
\end{flushleft}


\begin{flushleft}
\`{o} Type the following directly after the \ensuremath{\backslash}begin\{document\} command:
\end{flushleft}


\begin{flushleft}
\ensuremath{\backslash}title\{My First Document\}
\end{flushleft}


\begin{flushleft}
\ensuremath{\backslash}author\{My Name\}
\end{flushleft}


\begin{flushleft}
\ensuremath{\backslash}date\{\ensuremath{\backslash}today\}
\end{flushleft}


\begin{flushleft}
\ensuremath{\backslash}maketitle
\end{flushleft}


\begin{flushleft}
Your document should now look like figure 2.1.
\end{flushleft}


\begin{flushleft}
\`{o} Click on the Typeset button and check the PDF.
\end{flushleft}


\begin{flushleft}
Points to note:
\end{flushleft}


\begin{flushleft}
$\bullet$ \ensuremath{\backslash}today is a command that inserts today's date. You can also type in
\end{flushleft}


\begin{flushleft}
a different date, for example \ensuremath{\backslash}date\{November 2013\}.
\end{flushleft}


\begin{flushleft}
$\bullet$ Article documents start the text immediately below the title on the
\end{flushleft}


\begin{flushleft}
same page. Reports put the title on a separate page (like this workbook).
\end{flushleft}


5





\begin{flushleft}
\newpage
Figure 2.1: TeXworks screenshot showing the maketitle command.
\end{flushleft}





2.4





\begin{flushleft}
Sections
\end{flushleft}





\begin{flushleft}
You should divide your document into chapters (if needed), sections and subsections. The following sectioning commands are available for the article
\end{flushleft}


\begin{flushleft}
class:
\end{flushleft}


\begin{flushleft}
$\bullet$ \ensuremath{\backslash}section\{...\}
\end{flushleft}


\begin{flushleft}
$\bullet$ \ensuremath{\backslash}subsection\{...\}
\end{flushleft}


\begin{flushleft}
$\bullet$ \ensuremath{\backslash}subsubsection\{...\}
\end{flushleft}


\begin{flushleft}
$\bullet$ \ensuremath{\backslash}paragraph\{...\}
\end{flushleft}


\begin{flushleft}
$\bullet$ \ensuremath{\backslash}subparagraph\{...\}
\end{flushleft}


\begin{flushleft}
The title of the section replaces the dots between the curly brackets. With
\end{flushleft}


\begin{flushleft}
the report and book classes we also have \ensuremath{\backslash}chapter\{...\}.
\end{flushleft}


\begin{flushleft}
\`{o} Replace {``}A sentence of text.'' with the following:
\end{flushleft}


\begin{flushleft}
\ensuremath{\backslash}section\{Introduction\}
\end{flushleft}


\begin{flushleft}
This is the introduction.
\end{flushleft}


\begin{flushleft}
\ensuremath{\backslash}section\{Methods\}
\end{flushleft}


\begin{flushleft}
\ensuremath{\backslash}subsection\{Stage 1\}
\end{flushleft}


\begin{flushleft}
The first part of the methods.
\end{flushleft}





6





\begin{flushleft}
\newpage
\ensuremath{\backslash}subsection\{Stage 2\}
\end{flushleft}


\begin{flushleft}
The second part of the methods.
\end{flushleft}


\begin{flushleft}
\ensuremath{\backslash}section\{Results\}
\end{flushleft}


\begin{flushleft}
Here are my results.
\end{flushleft}


\begin{flushleft}
Your document should now look like figure 2.2.
\end{flushleft}





\begin{flushleft}
Figure 2.2: TeXworks screenshot of document with sections.
\end{flushleft}





\begin{flushleft}
\`{o} Click on the Typeset button and check the PDF.
\end{flushleft}





2.5





\begin{flushleft}
Labelling
\end{flushleft}





\begin{flushleft}
You can label any of the sectioning commands so they can be referred to in
\end{flushleft}


\begin{flushleft}
other parts of the document. Label the section with \ensuremath{\backslash}label\{labelname\}.
\end{flushleft}


\begin{flushleft}
Then type \ensuremath{\backslash}ref\{labelname\} or \ensuremath{\backslash}pageref\{labelname\}, when you want to
\end{flushleft}


\begin{flushleft}
refer to the section or page number of the label.
\end{flushleft}


\begin{flushleft}
\`{o} Type \ensuremath{\backslash}label\{sec1\} on a new line directly below \ensuremath{\backslash}subsection\{Stage 1\}.
\end{flushleft}


7





\begin{flushleft}
\newpage
\`{o} Type Referring to section \ensuremath{\backslash}ref\{sec1\} on page \ensuremath{\backslash}pageref\{sec1\}
\end{flushleft}


\begin{flushleft}
in the Results section.
\end{flushleft}


\begin{flushleft}
Your document should now look like figure 2.3.
\end{flushleft}





\begin{flushleft}
Figure 2.3: TeXworks screenshot of document with labels.
\end{flushleft}





\begin{flushleft}
\`{o} Click on the Typeset button and check the PDF. You may need to
\end{flushleft}


\begin{flushleft}
typeset the document twice before the references appear in the PDF.
\end{flushleft}





2.6





\begin{flushleft}
Table of Contents
\end{flushleft}





\begin{flushleft}
If you use sectioning commands it is very easy to generate a table of contents.
\end{flushleft}


\begin{flushleft}
Type \ensuremath{\backslash}tableofcontents where you want the table of contents to appear in
\end{flushleft}


\begin{flushleft}
your document --- often directly after the title page.
\end{flushleft}


\begin{flushleft}
You may also want to change the page numbering so that roman numerals
\end{flushleft}


\begin{flushleft}
(i, ii, iii) are used for pages before the main document starts. This will also
\end{flushleft}


\begin{flushleft}
ensure that the main document starts on page 1. Page numbering can be
\end{flushleft}


\begin{flushleft}
switched between arabic and roman using \ensuremath{\backslash}pagenumbering\{...\}.
\end{flushleft}


8





\begin{flushleft}
\newpage
\`{o} Type the following on a new line below \ensuremath{\backslash}maketitle:
\end{flushleft}


\begin{flushleft}
\ensuremath{\backslash}pagenumbering\{roman\}
\end{flushleft}


\begin{flushleft}
\ensuremath{\backslash}tableofcontents
\end{flushleft}


\begin{flushleft}
\ensuremath{\backslash}newpage
\end{flushleft}


\begin{flushleft}
\ensuremath{\backslash}pagenumbering\{arabic\}
\end{flushleft}


\begin{flushleft}
The \ensuremath{\backslash}newpage command inserts a page break so that we can see the effect of
\end{flushleft}


\begin{flushleft}
the page numbering commands. The first 14 lines of code should now look
\end{flushleft}


\begin{flushleft}
like figure 2.4.
\end{flushleft}





\begin{flushleft}
Figure 2.4: TeXworks screenshot of document showing table of contents command.
\end{flushleft}


\begin{flushleft}
\`{o} Click on the Typeset button and check the PDF.
\end{flushleft}





9





\newpage
10





\begin{flushleft}
\newpage
Chapter 3
\end{flushleft}


\begin{flushleft}
Typesetting Text
\end{flushleft}


3.1





\begin{flushleft}
Font Effects
\end{flushleft}





\begin{flushleft}
There are LATEX commands for a variety of font effects:
\end{flushleft}


\begin{flushleft}
\ensuremath{\backslash}textit\{words in italics\}
\end{flushleft}


\begin{flushleft}
\ensuremath{\backslash}textsl\{words slanted\}
\end{flushleft}


\begin{flushleft}
\ensuremath{\backslash}textsc\{words in smallcaps\}
\end{flushleft}


\begin{flushleft}
\ensuremath{\backslash}textbf\{words in bold\}
\end{flushleft}


\begin{flushleft}
\ensuremath{\backslash}texttt\{words in teletype\}
\end{flushleft}


\begin{flushleft}
\ensuremath{\backslash}textsf\{sans serif words\}
\end{flushleft}


\begin{flushleft}
\ensuremath{\backslash}textrm\{roman words\}
\end{flushleft}


\begin{flushleft}
\ensuremath{\backslash}underline\{underlined words\}
\end{flushleft}





\begin{flushleft}
words in italics
\end{flushleft}


\begin{flushleft}
words slanted
\end{flushleft}


\begin{flushleft}
words in smallcaps
\end{flushleft}


\begin{flushleft}
words in bold
\end{flushleft}


\begin{flushleft}
words in teletype
\end{flushleft}


\begin{flushleft}
sans serif words
\end{flushleft}


\begin{flushleft}
roman words
\end{flushleft}


\begin{flushleft}
underlined words
\end{flushleft}





\begin{flushleft}
\`{o} Add some more text to your document and experiment with different
\end{flushleft}


\begin{flushleft}
text effects.
\end{flushleft}





3.2





\begin{flushleft}
Coloured Text
\end{flushleft}





\begin{flushleft}
To put coloured text in your document you need to use a package. There
\end{flushleft}


\begin{flushleft}
are many packages that can be used with LATEX to enhance its functionality.
\end{flushleft}


\begin{flushleft}
Packages are included in the preamble (i.e. before the \ensuremath{\backslash}begin\{document\}
\end{flushleft}


\begin{flushleft}
command). Packages are activated using the \ensuremath{\backslash}usepackage[options]\{package\}
\end{flushleft}


\begin{flushleft}
command, where package is the name of the package and options is an optional list of keywords that trigger special features in the package.
\end{flushleft}


11





\begin{flushleft}
\newpage
The basic colour names that \ensuremath{\backslash}usepackage\{color\} knows about are black,
\end{flushleft}


\begin{flushleft}
red, green, blue, cyan, magenta, yellow and white:
\end{flushleft}


\begin{flushleft}
Red, green, blue, cyan, magenta, yellow and white .
\end{flushleft}


\begin{flushleft}
The following code to produces coloured text:
\end{flushleft}


\begin{flushleft}
\{\ensuremath{\backslash}color\{colour\_name\}text\}
\end{flushleft}


\begin{flushleft}
Where colour\_name is the name of the colour you want, and text is the text
\end{flushleft}


\begin{flushleft}
you want to be coloured.
\end{flushleft}


\begin{flushleft}
\`{o} Type \ensuremath{\backslash}usepackage\{color\} on the line before \ensuremath{\backslash}begin\{document\}.
\end{flushleft}


\begin{flushleft}
\`{o} Type \{\ensuremath{\backslash}color\{red\}fire\} in your document.
\end{flushleft}


\begin{flushleft}
\`{o} Click on the Typeset button and check the PDF.
\end{flushleft}


\begin{flushleft}
The word {`}fire' should appear in red.
\end{flushleft}


\begin{flushleft}
It is possible to add options that allow \ensuremath{\backslash}usepackage\{color\} to understand
\end{flushleft}


\begin{flushleft}
more colour names, and even to define your own colours. It is also possible
\end{flushleft}


\begin{flushleft}
to change the background colour of text (as for white and yellow in the
\end{flushleft}


\begin{flushleft}
example above), but this is beyond the scope of this workbook. If you want
\end{flushleft}


\begin{flushleft}
more information about see the Colors chapter in the LATEX Wikibook1 .
\end{flushleft}





3.3





\begin{flushleft}
Font Sizes
\end{flushleft}





\begin{flushleft}
There are LATEX commands for a range of font sizes:
\end{flushleft}


\begin{flushleft}
\{\ensuremath{\backslash}tiny tiny words\}
\end{flushleft}


\begin{flushleft}
\{\ensuremath{\backslash}scriptsize scriptsize words\}
\end{flushleft}


\begin{flushleft}
\{\ensuremath{\backslash}footnotesize footnotesize words\}
\end{flushleft}


\begin{flushleft}
\{\ensuremath{\backslash}small small words\}
\end{flushleft}


\begin{flushleft}
\{\ensuremath{\backslash}normalsize normalsize words\}
\end{flushleft}


\begin{flushleft}
\{\ensuremath{\backslash}large large words\}
\end{flushleft}


\begin{flushleft}
\{\ensuremath{\backslash}Large Large words\}
\end{flushleft}





1





\begin{flushleft}
tiny words
\end{flushleft}





\begin{flushleft}
scriptsize words
\end{flushleft}





\begin{flushleft}
footnotesize words
\end{flushleft}





\begin{flushleft}
small words
\end{flushleft}





\begin{flushleft}
normalsize words
\end{flushleft}





\begin{flushleft}
large words
\end{flushleft}





\begin{flushleft}
Large words
\end{flushleft}





\begin{flushleft}
\{\ensuremath{\backslash}LARGE LARGE words\}
\end{flushleft}





\begin{flushleft}
LARGE words
\end{flushleft}





\begin{flushleft}
\{\ensuremath{\backslash}huge huge words\}
\end{flushleft}





\begin{flushleft}
huge words
\end{flushleft}





\begin{flushleft}
http://en.wikibooks.org/wiki/LaTeX/Colors
\end{flushleft}





12





\begin{flushleft}
\newpage
\`{o} Experiment with different font sizes in your document.
\end{flushleft}





3.4





\begin{flushleft}
Lists
\end{flushleft}





\begin{flushleft}
LATEX supports two types of lists: enumerate produces numbered lists, while
\end{flushleft}


\begin{flushleft}
itemize is for bulleted lists. Each list item is defined by \ensuremath{\backslash}item. Lists can
\end{flushleft}


\begin{flushleft}
be nested to produce sub-lists.
\end{flushleft}


\begin{flushleft}
\`{o} Type the following to produce a numbered list with a bulleted sub-list:
\end{flushleft}


\begin{flushleft}
\ensuremath{\backslash}begin\{enumerate\}
\end{flushleft}


\begin{flushleft}
\ensuremath{\backslash}item First thing
\end{flushleft}


\begin{flushleft}
\ensuremath{\backslash}item Second thing
\end{flushleft}


\begin{flushleft}
\ensuremath{\backslash}begin\{itemize\}
\end{flushleft}


\begin{flushleft}
\ensuremath{\backslash}item A sub-thing
\end{flushleft}


\begin{flushleft}
\ensuremath{\backslash}item Another sub-thing
\end{flushleft}


\begin{flushleft}
\ensuremath{\backslash}end\{itemize\}
\end{flushleft}


\begin{flushleft}
\ensuremath{\backslash}item Third thing
\end{flushleft}


\begin{flushleft}
\ensuremath{\backslash}end\{enumerate\}
\end{flushleft}


\begin{flushleft}
\`{o} Click on the Typeset button and check the PDF.
\end{flushleft}


\begin{flushleft}
The list should look like this:
\end{flushleft}


\begin{flushleft}
1. First thing
\end{flushleft}


\begin{flushleft}
2. Second thing
\end{flushleft}


\begin{flushleft}
$\bullet$ A sub-thing
\end{flushleft}


\begin{flushleft}
$\bullet$ Another sub-thing
\end{flushleft}


\begin{flushleft}
3. Third thing
\end{flushleft}


\begin{flushleft}
It is easy to change the bullet symbol using square brackets after the \ensuremath{\backslash}item,
\end{flushleft}


\begin{flushleft}
for example, \ensuremath{\backslash}item[-] will give a dash as the bullet. You can even use words
\end{flushleft}


\begin{flushleft}
as bullets, for example, \ensuremath{\backslash}item[One].
\end{flushleft}


\begin{flushleft}
The following code:
\end{flushleft}


13





\begin{flushleft}
\newpage
\ensuremath{\backslash}begin\{itemize\}
\end{flushleft}


\begin{flushleft}
\ensuremath{\backslash}item[-] First thing
\end{flushleft}


\begin{flushleft}
\ensuremath{\backslash}item[+] Second thing
\end{flushleft}


\begin{flushleft}
\ensuremath{\backslash}begin\{itemize\}
\end{flushleft}


\begin{flushleft}
\ensuremath{\backslash}item[Fish] A sub-thing
\end{flushleft}


\begin{flushleft}
\ensuremath{\backslash}item[Plants] Another sub-thing
\end{flushleft}


\begin{flushleft}
\ensuremath{\backslash}end\{itemize\}
\end{flushleft}


\begin{flushleft}
\ensuremath{\backslash}item[Q] Third thing
\end{flushleft}


\begin{flushleft}
\ensuremath{\backslash}end\{itemize\}
\end{flushleft}


\begin{flushleft}
Produces:
\end{flushleft}


\begin{flushleft}
- First thing
\end{flushleft}


\begin{flushleft}
+ Second thing
\end{flushleft}


\begin{flushleft}
Fish A sub-thing
\end{flushleft}


\begin{flushleft}
Plants Another sub-thing
\end{flushleft}


\begin{flushleft}
Q Third thing
\end{flushleft}





3.5





\begin{flushleft}
Comments \& Spacing
\end{flushleft}





\begin{flushleft}
Comments are created using \%. When LATEX encounters a \% character while
\end{flushleft}


\begin{flushleft}
processing a .tex file, it ignores the rest of the line (until the [Return] key
\end{flushleft}


\begin{flushleft}
has been pressed to start a new line --- not to be confused with line wrapping
\end{flushleft}


\begin{flushleft}
in your editor). This can be used to write notes in the input file which will
\end{flushleft}


\begin{flushleft}
not show up in the printed version.
\end{flushleft}


\begin{flushleft}
The following code:
\end{flushleft}


\begin{flushleft}
It is a truth universally acknowledged\% Note comic irony
\end{flushleft}


\begin{flushleft}
in the very first sentence
\end{flushleft}


\begin{flushleft}
, that a single man in possession of a good fortune, must
\end{flushleft}


\begin{flushleft}
be in want of a wife.
\end{flushleft}


\begin{flushleft}
Produces:
\end{flushleft}


\begin{flushleft}
It is a truth universally acknowledged, that a single man in possession of a good fortune, must be in want of a wife.
\end{flushleft}


14





\begin{flushleft}
\newpage
Multiple consecutive spaces in LATEX are treated as a single space. Several
\end{flushleft}


\begin{flushleft}
empty lines are treated as one empty line. The main function of an empty
\end{flushleft}


\begin{flushleft}
line in LATEX is to start a new paragraph. In general, LATEX ignores blank
\end{flushleft}


\begin{flushleft}
lines and other empty space in the .tex file. Two backslashes (\ensuremath{\backslash}\ensuremath{\backslash}) can be
\end{flushleft}


\begin{flushleft}
used to start a new line.
\end{flushleft}


\begin{flushleft}
\`{o} Experiment with putting comments and blank lines in to your document.
\end{flushleft}


\begin{flushleft}
If you want to add blank space into your document use the \ensuremath{\backslash}vspace\{...\}
\end{flushleft}


\begin{flushleft}
command. This will add blank vertical space of a height specified in typographical points (pt). For example, \ensuremath{\backslash}vspace\{12pt\} will add space equivalent
\end{flushleft}


\begin{flushleft}
to the height of a 12pt font.
\end{flushleft}





3.6





\begin{flushleft}
Special Characters
\end{flushleft}





\begin{flushleft}
The following symbols are reserved characters which have a special meaning
\end{flushleft}


\begin{flushleft}
in LATEX:
\end{flushleft}


\#





\$





\%





\^{}





\&





\_





\{





\}





\~{}





\ensuremath{\backslash}





\begin{flushleft}
All of these apart from the backslash \ensuremath{\backslash} can be inserted as characters in your
\end{flushleft}


\begin{flushleft}
document by adding a prefix backslash:
\end{flushleft}


\ensuremath{\backslash}\#





\ensuremath{\backslash}\$





\ensuremath{\backslash}\%





\ensuremath{\backslash}\^{}\{\} \ensuremath{\backslash}\&





\ensuremath{\backslash}\_





\ensuremath{\backslash}\{





\ensuremath{\backslash}\}





\ensuremath{\backslash}\~{}\{\}





\begin{flushleft}
Note that you need to type a pair of curly brackets \{\} after the hat \^{} and
\end{flushleft}


\begin{flushleft}
tilde \~{}, otherwise these will appear as accents over the following character.
\end{flushleft}


\begin{flushleft}
For example, {``}\ensuremath{\backslash}\^{} e'' produces {``}ê''.
\end{flushleft}


\begin{flushleft}
The above code will produce:
\end{flushleft}


\#





\$





\%





ˆ





\&





\{





\}





˜





\begin{flushleft}
The backslash character \ensuremath{\backslash} can not be entered by adding a prefix backslash,
\end{flushleft}


\begin{flushleft}
\ensuremath{\backslash}\ensuremath{\backslash}, as this is used for line breaking. Use the \ensuremath{\backslash}textbackslash command
\end{flushleft}


\begin{flushleft}
instead.
\end{flushleft}


\begin{flushleft}
\`{o} Type code to produce the following sentence in your document:
\end{flushleft}


\begin{flushleft}
Item \#1A\ensuremath{\backslash}642 costs \$8 \& is sold at a ˜10\% profit.
\end{flushleft}


\begin{flushleft}
Ask the tutor, or check the .tex file of this workbook, if you need help.
\end{flushleft}


15





\newpage
16





\begin{flushleft}
\newpage
Chapter 4
\end{flushleft}


\begin{flushleft}
Tables
\end{flushleft}


\begin{flushleft}
The tabular command is used to typeset tables. By default, LATEX tables
\end{flushleft}


\begin{flushleft}
are drawn without horizontal and vertical lines --- you need to specify if you
\end{flushleft}


\begin{flushleft}
want lines drawn. LATEX determines the width of the columns automatically.
\end{flushleft}


\begin{flushleft}
This code starts a table:
\end{flushleft}


\begin{flushleft}
\ensuremath{\backslash}begin\{tabular\}\{...\}
\end{flushleft}


\begin{flushleft}
Where the dots between the curly brackets are replaced by code defining the
\end{flushleft}


\begin{flushleft}
columns:
\end{flushleft}


\begin{flushleft}
$\bullet$ l for a column of left-aligned text (letter el, not number one).
\end{flushleft}


\begin{flushleft}
$\bullet$ r for a column of right-aligned text.
\end{flushleft}


\begin{flushleft}
$\bullet$ c for a column of centre-aligned text.
\end{flushleft}


\begin{flushleft}
$\bullet$ | for a vertical line.
\end{flushleft}


\begin{flushleft}
For example, \{lll\} (i.e. left left left) will produce 3 columns of left-aligned
\end{flushleft}


\begin{flushleft}
text with no vertical lines , while \{|l|l|r|\} (i.e. |left|left|right|) will produce
\end{flushleft}


\begin{flushleft}
3 columns --- the first 2 are left-aligned, the third is right-aligned, and there
\end{flushleft}


\begin{flushleft}
are vertical lines around each column.
\end{flushleft}


\begin{flushleft}
The table data follows the \ensuremath{\backslash}begin command:
\end{flushleft}


\begin{flushleft}
$\bullet$ \& is placed between columns.
\end{flushleft}


\begin{flushleft}
$\bullet$ \ensuremath{\backslash}\ensuremath{\backslash} is placed at the end of a row (to start a new one).
\end{flushleft}


17





\begin{flushleft}
\newpage
$\bullet$ \ensuremath{\backslash}hline inserts a horizontal line.
\end{flushleft}


\begin{flushleft}
$\bullet$ \ensuremath{\backslash}cline\{1-2\} inserts a partial horizontal line between column 1 and
\end{flushleft}


\begin{flushleft}
column 2.
\end{flushleft}


\begin{flushleft}
The command \ensuremath{\backslash}end\{tabular\} finishes the table.
\end{flushleft}


\begin{flushleft}
Examples of tabular code and the resulting tables:
\end{flushleft}


\begin{flushleft}
\ensuremath{\backslash}begin\{tabular\}\{|l|l|\}
\end{flushleft}


\begin{flushleft}
Apples \& Green \ensuremath{\backslash}\ensuremath{\backslash}
\end{flushleft}


\begin{flushleft}
Strawberries \& Red \ensuremath{\backslash}\ensuremath{\backslash}
\end{flushleft}


\begin{flushleft}
Oranges \& Orange \ensuremath{\backslash}\ensuremath{\backslash}
\end{flushleft}


\begin{flushleft}
\ensuremath{\backslash}end\{tabular\}
\end{flushleft}





\begin{flushleft}
Apples
\end{flushleft}


\begin{flushleft}
Strawberries
\end{flushleft}


\begin{flushleft}
Oranges
\end{flushleft}





\begin{flushleft}
\ensuremath{\backslash}begin\{tabular\}\{rc\}
\end{flushleft}


\begin{flushleft}
Apples \& Green \ensuremath{\backslash}\ensuremath{\backslash}
\end{flushleft}


\begin{flushleft}
\ensuremath{\backslash}hline
\end{flushleft}


\begin{flushleft}
Strawberries \& Red \ensuremath{\backslash}\ensuremath{\backslash}
\end{flushleft}


\begin{flushleft}
\ensuremath{\backslash}cline\{1-1\}
\end{flushleft}


\begin{flushleft}
Oranges \& Orange \ensuremath{\backslash}\ensuremath{\backslash}
\end{flushleft}


\begin{flushleft}
\ensuremath{\backslash}end\{tabular\}
\end{flushleft}





\begin{flushleft}
Apples Green
\end{flushleft}


\begin{flushleft}
Strawberries
\end{flushleft}


\begin{flushleft}
Red
\end{flushleft}


\begin{flushleft}
Oranges Orange
\end{flushleft}





\begin{flushleft}
\ensuremath{\backslash}begin\{tabular\}\{|r|l|\}
\end{flushleft}


\begin{flushleft}
\ensuremath{\backslash}hline
\end{flushleft}


\begin{flushleft}
8 \& here's \ensuremath{\backslash}\ensuremath{\backslash}
\end{flushleft}


\begin{flushleft}
\ensuremath{\backslash}cline\{2-2\}
\end{flushleft}


\begin{flushleft}
86 \& stuff \ensuremath{\backslash}\ensuremath{\backslash}
\end{flushleft}


\begin{flushleft}
\ensuremath{\backslash}hline \ensuremath{\backslash}hline
\end{flushleft}


\begin{flushleft}
2008 \& now \ensuremath{\backslash}\ensuremath{\backslash}
\end{flushleft}


\begin{flushleft}
\ensuremath{\backslash}hline
\end{flushleft}


\begin{flushleft}
\ensuremath{\backslash}end\{tabular\}
\end{flushleft}





8


86


2008





4.1





\begin{flushleft}
Practical
\end{flushleft}





\begin{flushleft}
\`{o} Write code to produce the following tables:
\end{flushleft}


\begin{flushleft}
Item
\end{flushleft}


\begin{flushleft}
Nails
\end{flushleft}


\begin{flushleft}
Wooden boards
\end{flushleft}


\begin{flushleft}
Bricks
\end{flushleft}





\begin{flushleft}
Quantity
\end{flushleft}


500


100


240





\begin{flushleft}
Price (\$)
\end{flushleft}


0.34


4.00


11.50


18





\begin{flushleft}
here's
\end{flushleft}


\begin{flushleft}
stuff
\end{flushleft}


\begin{flushleft}
now
\end{flushleft}





\begin{flushleft}
Green
\end{flushleft}


\begin{flushleft}
Red
\end{flushleft}


\begin{flushleft}
Orange
\end{flushleft}





\begin{flushleft}
\newpage
City
\end{flushleft}


\begin{flushleft}
London
\end{flushleft}


\begin{flushleft}
Berlin
\end{flushleft}


\begin{flushleft}
Paris
\end{flushleft}





2006


45789


34549


49835





\begin{flushleft}
Year
\end{flushleft}


2007


46551


32543


51009





2008


51298


29870


51970





\begin{flushleft}
Ask the tutor, or look at the .tex file of this workbook, if you need help.
\end{flushleft}





19





\newpage
20





\begin{flushleft}
\newpage
Chapter 5
\end{flushleft}


\begin{flushleft}
Figures
\end{flushleft}


\begin{flushleft}
This chapter describes how to insert an image in to your LATEX document,
\end{flushleft}


\begin{flushleft}
which requires the graphicx package. Images should be PDF, PNG, JPEG
\end{flushleft}


\begin{flushleft}
or GIF files. The following code will insert an image called myimage:
\end{flushleft}





\begin{flushleft}
\ensuremath{\backslash}begin\{figure\}[h]
\end{flushleft}


\begin{flushleft}
\ensuremath{\backslash}centering
\end{flushleft}


\begin{flushleft}
\ensuremath{\backslash}includegraphics[width=1\ensuremath{\backslash}textwidth]\{myimage\}
\end{flushleft}


\begin{flushleft}
\ensuremath{\backslash}caption\{Here is my image\}
\end{flushleft}


\begin{flushleft}
\ensuremath{\backslash}label\{image-myimage\}
\end{flushleft}


\begin{flushleft}
\ensuremath{\backslash}end\{figure\}
\end{flushleft}





\begin{flushleft}
[h] is the placement specifier. h means put the figure approximately here (if
\end{flushleft}


\begin{flushleft}
it will fit). Other options are t (at the top of the page), b (at the bottom of
\end{flushleft}


\begin{flushleft}
the page) and p (on a separate page for figures). You can also add !, which
\end{flushleft}


\begin{flushleft}
overrides the rule LATEX uses for choosing where to put the figure, and makes
\end{flushleft}


\begin{flushleft}
it more likely it will put it where you want (even if it doesn't look so good).
\end{flushleft}


\begin{flushleft}
\ensuremath{\backslash}centering centres the image on the page, if not used images are left-aligned
\end{flushleft}


\begin{flushleft}
by default. It's a good idea to use this as the figure captions are centred.
\end{flushleft}


\begin{flushleft}
includegraphics\{...\} is the command that actually puts the image in your
\end{flushleft}


\begin{flushleft}
document. The image file should be saved in the same folder as the .tex file.
\end{flushleft}


\begin{flushleft}
[width=1\ensuremath{\backslash}textwidth] is an optional command that specifies the width of
\end{flushleft}


\begin{flushleft}
the picture - in this case the same width as the text. The width could also
\end{flushleft}


\begin{flushleft}
be given in centimeters (cm). You could also use [scale=0.5] which scales
\end{flushleft}


\begin{flushleft}
the image by the desired factor, in this case reducing by half.
\end{flushleft}


21





\begin{flushleft}
\newpage
\ensuremath{\backslash}caption\{...\} defines a caption for the figure. If this is used LATEX will add
\end{flushleft}


\begin{flushleft}
{``}Figure'' and a number before the caption. If you use captions, you can use
\end{flushleft}


\begin{flushleft}
\ensuremath{\backslash}listoffigures to create a table of figures in a similar way to the table of
\end{flushleft}


\begin{flushleft}
contents (section 2.6, page 8).
\end{flushleft}


\begin{flushleft}
\ensuremath{\backslash}label\{...\} creates a label to allow you to refer to the table or figure in
\end{flushleft}


\begin{flushleft}
your text (section 2.5, page 7).
\end{flushleft}





5.1





\begin{flushleft}
Practical
\end{flushleft}





\begin{flushleft}
\`{o} Add \ensuremath{\backslash}usepackage\{graphicx\} in the preamble of your document (before the \ensuremath{\backslash}begin\{document\} command).
\end{flushleft}


\begin{flushleft}
\`{o} Find an image and save a copy to your LaTeX course folder.
\end{flushleft}


\begin{flushleft}
\`{o} Type the following text at the point where you want your image inserted:
\end{flushleft}


\begin{flushleft}
\ensuremath{\backslash}begin\{figure\}[h!]
\end{flushleft}


\begin{flushleft}
\ensuremath{\backslash}centering
\end{flushleft}


\begin{flushleft}
\ensuremath{\backslash}includegraphics[width=1\ensuremath{\backslash}textwidth]\{ImageFilename\}
\end{flushleft}


\begin{flushleft}
\ensuremath{\backslash}caption\{My test image\}
\end{flushleft}


\begin{flushleft}
\ensuremath{\backslash}end\{figure\}
\end{flushleft}


\begin{flushleft}
Replace ImageFilename with the name of your image file, excluding the file
\end{flushleft}


\begin{flushleft}
extension. If there are any spaces in the file name enclose it in quotation
\end{flushleft}


\begin{flushleft}
marks, for example {``}screen 20''.
\end{flushleft}


\begin{flushleft}
\`{o} Click on the Typeset button and check the PDF.
\end{flushleft}





22





\begin{flushleft}
\newpage
Chapter 6
\end{flushleft}


\begin{flushleft}
Equations
\end{flushleft}


\begin{flushleft}
One of the main reasons for writing documents in LATEX is because it is really
\end{flushleft}


\begin{flushleft}
good at typesetting equations. Equations are written in {`}math mode'.
\end{flushleft}





6.1





\begin{flushleft}
Inserting Equations
\end{flushleft}





\begin{flushleft}
You can enter math mode with an opening and closing dollar sign \$. This
\end{flushleft}


\begin{flushleft}
can be used to write mathematical symbols within a sentence --- for example,
\end{flushleft}


\begin{flushleft}
typing \$1+2=3\$ produces 1 + 2 = 3.
\end{flushleft}


\begin{flushleft}
If you want a {``}displayed'' equation on its own line use \$\$...\$\$.
\end{flushleft}


\begin{flushleft}
For example, \$\$1+2=3\$\$ produces:
\end{flushleft}


1+2=3


\begin{flushleft}
For a numbered displayed equation, use \ensuremath{\backslash}begin\{equation\}...\ensuremath{\backslash}end\{equation\}.
\end{flushleft}


\begin{flushleft}
For example, \ensuremath{\backslash}begin\{equation\}1+2=3\ensuremath{\backslash}end\{equation\} produces:
\end{flushleft}


1+2=3





(6.1)





\begin{flushleft}
The number 6 refers to the chapter number, this will only appear if you are
\end{flushleft}


\begin{flushleft}
using a document class with chapters, such as report.
\end{flushleft}


\begin{flushleft}
Use \ensuremath{\backslash}begin\{eqnarray\}...\ensuremath{\backslash}end\{eqnarray\} to write equation arrays for a
\end{flushleft}


\begin{flushleft}
series of equations/inequalities. For example ---
\end{flushleft}


\begin{flushleft}
\ensuremath{\backslash}begin\{eqnarray\}
\end{flushleft}


23





\begin{flushleft}
\newpage
a \& = \& b + c \ensuremath{\backslash}\ensuremath{\backslash}
\end{flushleft}


\begin{flushleft}
\& = \& y - z
\end{flushleft}


\begin{flushleft}
\ensuremath{\backslash}end\{eqnarray\}
\end{flushleft}


\begin{flushleft}
Produces:
\end{flushleft}





\begin{flushleft}
a = b+c
\end{flushleft}


\begin{flushleft}
= y$-$z
\end{flushleft}





(6.2)


(6.3)





\begin{flushleft}
For unnumbered equations add the star symbol * after the equation or
\end{flushleft}


\begin{flushleft}
eqnarray command (i.e. use \{equation*\} or \{eqnarray*\}).
\end{flushleft}





6.2





\begin{flushleft}
Mathematical Symbols
\end{flushleft}





\begin{flushleft}
Although some basic mathematical symbols (+ - = ! / ( ) [ ] :) can be accessed
\end{flushleft}


\begin{flushleft}
directly from the keyboard, most must be inserted using a command.
\end{flushleft}


\begin{flushleft}
This section is a very brief introduction to using LATEX to produce mathematical symbols --- the Mathematics chapter in the LATEX Wikibook is an
\end{flushleft}


\begin{flushleft}
excellent tutorial on mathematical symbol commands, which you should refer
\end{flushleft}


\begin{flushleft}
to if you want to learn more. If you want to find the command for a specific
\end{flushleft}


\begin{flushleft}
symbol try Detexify1 , which can recognise hand drawn symbols.
\end{flushleft}





6.2.1





\begin{flushleft}
Powers \& Indices
\end{flushleft}





\begin{flushleft}
Powers are inserted using the hat \^{} symbol. For example, \$n\^{}2\$ produces
\end{flushleft}


\begin{flushleft}
n2 .
\end{flushleft}


\begin{flushleft}
Indices are inserted using an underscore \_. For example, \$2\_a\$ produces 2a .
\end{flushleft}


\begin{flushleft}
If the power or index includes more than one character, group them using
\end{flushleft}


\begin{flushleft}
curly brackets \{...\}, e.g. \$b\_\{a-2\}\$ produces ba$-$2 .
\end{flushleft}





6.2.2





\begin{flushleft}
Fractions
\end{flushleft}





\begin{flushleft}
Fractions are inserted using \ensuremath{\backslash}frac\{numerator\}\{denominator\}.
\end{flushleft}


1





\begin{flushleft}
http://detexify.kirelabs.org
\end{flushleft}





24





\begin{flushleft}
\newpage
\$\$\ensuremath{\backslash}frac\{a\}\{3\}\$\$ produces:
\end{flushleft}





\begin{flushleft}
a
\end{flushleft}


3





\begin{flushleft}
Fractions can be nested ---
\end{flushleft}


\begin{flushleft}
\$\$\ensuremath{\backslash}frac\{y\}\{\ensuremath{\backslash}frac\{3\}\{x\}+b\}\$\$ produces:
\end{flushleft}


3


\begin{flushleft}
x
\end{flushleft}





6.2.3





\begin{flushleft}
y
\end{flushleft}


\begin{flushleft}
+b
\end{flushleft}





\begin{flushleft}
Roots
\end{flushleft}





\begin{flushleft}
Square root symbols are inserted using \ensuremath{\backslash}sqrt\{...\} where ... is replaced by
\end{flushleft}


\begin{flushleft}
the square root content. If a magnitude is required it can be added using
\end{flushleft}


\begin{flushleft}
optional square brackets [...].
\end{flushleft}


\begin{flushleft}
\$\$\ensuremath{\backslash}sqrt\{y\^{}2\}\$\$ produces:
\end{flushleft}


\begin{flushleft}
q
\end{flushleft}





\begin{flushleft}
y2
\end{flushleft}





\begin{flushleft}
\$\$\ensuremath{\backslash}sqrt[x]\{y\^{}2\}\$\$ produces:
\end{flushleft}


\begin{flushleft}
q
\end{flushleft}


\begin{flushleft}
x
\end{flushleft}





6.2.4





\begin{flushleft}
y2
\end{flushleft}





\begin{flushleft}
Sums \& Integrals
\end{flushleft}





\begin{flushleft}
The command \ensuremath{\backslash}sum inserts a sum symbol; \ensuremath{\backslash}int inserts an integral. For
\end{flushleft}


\begin{flushleft}
both functions, the upper limit is specified by a hat ˆ and the lower by an
\end{flushleft}


\begin{flushleft}
underscore .
\end{flushleft}


\begin{flushleft}
\$\$\ensuremath{\backslash}sum\_\{x=1\}\^{}5 y\^{}z\$\$ produces:
\end{flushleft}


5


\begin{flushleft}
X
\end{flushleft}





\begin{flushleft}
yz
\end{flushleft}





\begin{flushleft}
x=1
\end{flushleft}





\begin{flushleft}
\$\$\ensuremath{\backslash}int\_a\^{}b f(x)\$\$ produces:
\end{flushleft}


\begin{flushleft}
Z b
\end{flushleft}





\begin{flushleft}
f (x)
\end{flushleft}





\begin{flushleft}
a
\end{flushleft}





6.2.5





\begin{flushleft}
Greek letters
\end{flushleft}





\begin{flushleft}
Greek letters can be typed in math mode using the name of the letter preceded by a backslash \ensuremath{\backslash}. Many Greek capital letters are used in the Latin
\end{flushleft}


25





\begin{flushleft}
\newpage
alphabet --- for those that are different capitalise the first letter of the name
\end{flushleft}


\begin{flushleft}
to produce a capital Greek letter.
\end{flushleft}


\begin{flushleft}
For example ---
\end{flushleft}


\begin{flushleft}
\$\ensuremath{\backslash}alpha\$ = $\alpha$
\end{flushleft}


\begin{flushleft}
\$\ensuremath{\backslash}beta\$ = $\beta$
\end{flushleft}


\begin{flushleft}
\$\ensuremath{\backslash}delta, \ensuremath{\backslash}Delta\$ = $\delta$, ∆
\end{flushleft}


\begin{flushleft}
\$\ensuremath{\backslash}theta, \ensuremath{\backslash}Theta\$ = $\theta$, $\Theta$
\end{flushleft}


\begin{flushleft}
\$\ensuremath{\backslash}mu\$ = $\mu$
\end{flushleft}


\begin{flushleft}
\$\ensuremath{\backslash}pi, \ensuremath{\backslash}Pi\$ = $\pi$, $\Pi$
\end{flushleft}


\begin{flushleft}
\$\ensuremath{\backslash}sigma, \ensuremath{\backslash}Sigma\$ = $\sigma$, $\Sigma$
\end{flushleft}


\begin{flushleft}
\$\ensuremath{\backslash}phi, \ensuremath{\backslash}Phi\$ = $\phi$, $\Phi$
\end{flushleft}


\begin{flushleft}
\$\ensuremath{\backslash}psi, \ensuremath{\backslash}Psi\$ = $\psi$, $\Psi$
\end{flushleft}


\begin{flushleft}
\$\ensuremath{\backslash}omega, \ensuremath{\backslash}Omega\$ = $\omega$, Ω
\end{flushleft}





6.3





\begin{flushleft}
Practical
\end{flushleft}





\begin{flushleft}
\`{o} Write code to produce the following equations:
\end{flushleft}


\begin{flushleft}
e = mc2
\end{flushleft}


\begin{flushleft}
$\pi$=
\end{flushleft}





(6.1)





\begin{flushleft}
c
\end{flushleft}


\begin{flushleft}
d
\end{flushleft}





(6.2)





\begin{flushleft}
d x
\end{flushleft}


\begin{flushleft}
e = ex
\end{flushleft}


\begin{flushleft}
dx
\end{flushleft}





(6.3)





\begin{flushleft}
d Z$\infty$
\end{flushleft}


\begin{flushleft}
f (s)ds = f (x)
\end{flushleft}


\begin{flushleft}
dx 0
\end{flushleft}





(6.4)





\begin{flushleft}
f (x) =
\end{flushleft}





\begin{flushleft}
X
\end{flushleft}





= 0$\infty$





\begin{flushleft}
i
\end{flushleft}





\begin{flushleft}
r
\end{flushleft}





\begin{flushleft}
x=
\end{flushleft}





\begin{flushleft}
f (i) (0) i
\end{flushleft}


\begin{flushleft}
x
\end{flushleft}


\begin{flushleft}
i!
\end{flushleft}





\begin{flushleft}
xi
\end{flushleft}


\begin{flushleft}
y
\end{flushleft}


\begin{flushleft}
z
\end{flushleft}





(6.5)





(6.6)





\begin{flushleft}
Ask the tutor, or look at the .tex file of this workbook, if you need help.
\end{flushleft}


26





\begin{flushleft}
\newpage
Chapter 7
\end{flushleft}


\begin{flushleft}
Inserting References
\end{flushleft}


7.1





\begin{flushleft}
Introduction
\end{flushleft}





\begin{flushleft}
LATEX includes features that allow you to easily cite references and create
\end{flushleft}


\begin{flushleft}
bibliographies in your document. This document will explain how to do this
\end{flushleft}


\begin{flushleft}
using a separate BibTeX file to store the details of your references.
\end{flushleft}





7.2





\begin{flushleft}
The BibTeX file
\end{flushleft}





\begin{flushleft}
Your BibTeX file contains all the references you want to cite in your document. It has the file extension .bib. It should be given the same name as and
\end{flushleft}


\begin{flushleft}
kept in the same folder as your .tex file. The .bib file is plain text - it can
\end{flushleft}


\begin{flushleft}
be edited using Notepad or your LATEX editor (e.g. TeXworks). You should
\end{flushleft}


\begin{flushleft}
enter each of your references in the BibTeX file in the following format:
\end{flushleft}





\begin{flushleft}
@article\{
\end{flushleft}


\begin{flushleft}
Birdetal2001,
\end{flushleft}


\begin{flushleft}
Author = \{Bird, R. B. and Smith, E. A. and Bird, D. W.\},
\end{flushleft}


\begin{flushleft}
Title = \{The hunting handicap: costly signaling in human
\end{flushleft}


\begin{flushleft}
foraging strategies\},
\end{flushleft}


\begin{flushleft}
Journal = \{Behavioral Ecology and Sociobiology\},
\end{flushleft}


\begin{flushleft}
Volume = \{50\},
\end{flushleft}


\begin{flushleft}
Pages = \{9-19\},
\end{flushleft}


\begin{flushleft}
Year = \{2001\} \}
\end{flushleft}


27





\begin{flushleft}
\newpage
Each reference starts with the reference type (@article in the example
\end{flushleft}


\begin{flushleft}
above). Other reference types include @book, @incollection for a chapter in
\end{flushleft}


\begin{flushleft}
an edited book and @inproceedings for papers presented at conferences1 .
\end{flushleft}


\begin{flushleft}
The reference type declaration is followed by a curly bracket, then the citation key. Each reference's citation key must be unique - you can use
\end{flushleft}


\begin{flushleft}
anything you want, but a system based on the first author's name and year
\end{flushleft}


\begin{flushleft}
(as in the example above) is probably easiest to keep track of.
\end{flushleft}


\begin{flushleft}
The remaining lines contain the reference information in the format
\end{flushleft}


\begin{flushleft}
Field name = \{field contents\},.
\end{flushleft}


\begin{flushleft}
You need to include LaTeX commands in your BibTeX file for any special
\end{flushleft}


\begin{flushleft}
text formatting - e.g. italics (\ensuremath{\backslash}emph\{Rattus norvegicus\}), quotation marks
\end{flushleft}


\begin{flushleft}
({`}{`}...''), ampersand (\ensuremath{\backslash}\&).
\end{flushleft}


\begin{flushleft}
Surround any letters in a journal article title that need to be capitalised
\end{flushleft}


\begin{flushleft}
with curly brackets \{...\}. BibTeX automatically uncapitalises any capital
\end{flushleft}


\begin{flushleft}
letters within the journal article title. For example, {``}Dispersal in the contemporary United States'' will be printed as {``}Dispersal in the contemporary
\end{flushleft}


\begin{flushleft}
united states'', but {``}Dispersal in the contemporary \{U\}nited \{S\}tates'' will
\end{flushleft}


\begin{flushleft}
be printed as {``}Dispersal in the contemporary United States''.
\end{flushleft}


\begin{flushleft}
You can type the BibTeX file yourself, or you can use reference management
\end{flushleft}


\begin{flushleft}
software such as EndNote to create it 2 .
\end{flushleft}





7.3





\begin{flushleft}
Inserting the bibliography
\end{flushleft}





\begin{flushleft}
Type the following where you want the bibliography to appear in your document (usually at the end):
\end{flushleft}


\begin{flushleft}
\ensuremath{\backslash}bibliographystyle\{plain\}
\end{flushleft}


\begin{flushleft}
\ensuremath{\backslash}bibliography\{Doc1\}
\end{flushleft}


\begin{flushleft}
Where references is the name of your .bib file.
\end{flushleft}





1





\begin{flushleft}
See the Bibliography Management chapter in the LaTeX Wikibook http://en.
\end{flushleft}


\begin{flushleft}
wikibooks.org/wiki/LaTeX/Bibliography\_Management for a full list of the reference
\end{flushleft}


\begin{flushleft}
types that BibTeX knows about, and their required and optional fields.
\end{flushleft}


2


\begin{flushleft}
Instructions for using EndNote with LaTeX are available on the HowTo wiki https:
\end{flushleft}


\begin{flushleft}
//www.wiki.ed.ac.uk/x/sZpKBg
\end{flushleft}





28





\newpage
7.4





\begin{flushleft}
Citing references
\end{flushleft}





\begin{flushleft}
Type \ensuremath{\backslash}cite\{citationkey\} where you want to cite a reference in your .tex
\end{flushleft}


\begin{flushleft}
document. If you don't want an in text citation, but still want the reference
\end{flushleft}


\begin{flushleft}
to appear in the bibliography, use \ensuremath{\backslash}nocite\{citationkey\}.
\end{flushleft}


\begin{flushleft}
To include a page number in your in-text citation put it in square brackets
\end{flushleft}


\begin{flushleft}
before the citation key: \ensuremath{\backslash}cite[p. 215]\{citationkey\}.
\end{flushleft}


\begin{flushleft}
To cite multiple references include all the citation keys within the curly brackets separated by commas: \ensuremath{\backslash}cite\{citation01,citation02,citation03\}.
\end{flushleft}





7.5


7.5.1





\begin{flushleft}
Styles
\end{flushleft}


\begin{flushleft}
Numerical citations
\end{flushleft}





\begin{flushleft}
LATEX comes with several styles with numerical in-text citations, these include:
\end{flushleft}


\begin{flushleft}
Plain The citation is a number in square brackets (e.g. [1]). The bibliography is ordered alphabetically by first author surname. All of the authors'
\end{flushleft}


\begin{flushleft}
names are written in full.
\end{flushleft}


\begin{flushleft}
Abbrv The same as plain except the authors' first names are abbreviated
\end{flushleft}


\begin{flushleft}
to an initial.
\end{flushleft}


\begin{flushleft}
Unsrt The same as plain except the references in the bibliography appear
\end{flushleft}


\begin{flushleft}
in the order that the citations appear in the document.
\end{flushleft}


\begin{flushleft}
Alpha The same as plain except the citation is an alphanumeric abbreviation based on the author(s) surname(s) and year of publication, surrounded
\end{flushleft}


\begin{flushleft}
by square brackets (e.g. [Kop10]).
\end{flushleft}





7.5.2





\begin{flushleft}
Author-date citations
\end{flushleft}





\begin{flushleft}
Use the natbib package if you want to include author-date citations. Natbib
\end{flushleft}


\begin{flushleft}
uses the command \ensuremath{\backslash}citep\{...\} for a citation in brackets (e.g. [Koppe,
\end{flushleft}


29





\begin{flushleft}
\newpage
2010]) and \ensuremath{\backslash}citet\{...\} for a citation where only the year is in brackets
\end{flushleft}


\begin{flushleft}
(e.g. Koppe [2010]). There are lots of other ways that you can modify
\end{flushleft}


\begin{flushleft}
citations when using the natbib package - see the package's reference sheet
\end{flushleft}


\begin{flushleft}
for full details3 .
\end{flushleft}


\begin{flushleft}
Natbib comes with three bibliography styles: plainnat, abbrvnat and unsrtnat. These format the bibliography in the same way as the plain, abbrv
\end{flushleft}


\begin{flushleft}
and unsrt styles, respectively.
\end{flushleft}





7.5.3





\begin{flushleft}
Other bibliography styles
\end{flushleft}





\begin{flushleft}
If you want to use a different style (e.g. one provided by the journal you
\end{flushleft}


\begin{flushleft}
are submitting an article to) you should save the style file (.bst file) in the
\end{flushleft}


\begin{flushleft}
same folder as your .tex and .bib files. Include the name of the .bst file in
\end{flushleft}


\begin{flushleft}
the \ensuremath{\backslash}bibliographystyle\{...\} commmand.
\end{flushleft}





7.6





\begin{flushleft}
Practical
\end{flushleft}





\begin{flushleft}
\`{o} Create a new file in TeXworks (File menu $>$ New).
\end{flushleft}


\begin{flushleft}
\`{o} Type your references in the correct format (see example at start of
\end{flushleft}


\begin{flushleft}
chapter).
\end{flushleft}


\begin{flushleft}
\`{o} Click the Save button, the Save File window will open.
\end{flushleft}


\begin{flushleft}
\`{o} Give the file the same name as your .tex document (for example, Doc1)
\end{flushleft}


\begin{flushleft}
and save it as a BibTeX database in the same folder as your .tex file.
\end{flushleft}


\begin{flushleft}
\`{o} Switch to your .tex document and insert \ensuremath{\backslash}cite, \ensuremath{\backslash}bibliographystyle
\end{flushleft}


\begin{flushleft}
and \ensuremath{\backslash}bibliography commands in the relevant places.
\end{flushleft}


\begin{flushleft}
\`{o} Typeset your .tex file.
\end{flushleft}


\begin{flushleft}
\`{o} Switch to your .bib file, choose BibTeX from the typeset menu and
\end{flushleft}


\begin{flushleft}
click the Typeset button.
\end{flushleft}


\begin{flushleft}
\`{o} Switch to your .tex file and typeset it twice. The in-text citations and
\end{flushleft}


\begin{flushleft}
reference list should be inserted.
\end{flushleft}





3





\begin{flushleft}
Reference sheet for natbib usage http://mirror.ctan.org/macros/latex/contrib/
\end{flushleft}


\begin{flushleft}
natbib/natnotes.pdf
\end{flushleft}





30





\begin{flushleft}
\newpage
Chapter 8
\end{flushleft}


\begin{flushleft}
Further Reading
\end{flushleft}


\begin{flushleft}
LATEX Project
\end{flushleft}


\begin{flushleft}
http://www.latex-project.org/
\end{flushleft}


\begin{flushleft}
Official website - has links to documentation, information about installing
\end{flushleft}


\begin{flushleft}
LATEX on your own computer, and information about where to look for help.
\end{flushleft}


\begin{flushleft}
The Not So Short Introduction to LATEX2e
\end{flushleft}


\begin{flushleft}
http://ctan.tug.org/tex-archive/info/lshort/english/lshort.pdf
\end{flushleft}


\begin{flushleft}
A good tutorial for beginners.
\end{flushleft}


\begin{flushleft}
LATEX Wikibook
\end{flushleft}


\begin{flushleft}
http://en.wikibooks.org/wiki/LaTeX/
\end{flushleft}


\begin{flushleft}
Comprehensive and clearly written, although still a work in progress. A
\end{flushleft}


\begin{flushleft}
downloadable PDF is also available.
\end{flushleft}


\begin{flushleft}
Comparison of TEX Editors on Wikipedia
\end{flushleft}


\begin{flushleft}
http://en.wikipedia.org/wiki/Comparison\_of\_TeX\_editors
\end{flushleft}


\begin{flushleft}
Information to help you to choose which LATEX editor to install on your own
\end{flushleft}


\begin{flushleft}
computer.
\end{flushleft}


\begin{flushleft}
TeX Live
\end{flushleft}


\begin{flushleft}
http://www.tug.org/texlive/
\end{flushleft}


\begin{flushleft}
{``}An easy way to get up and running with the TeX document production
\end{flushleft}


\begin{flushleft}
system''. Available for Unix and Windows (links to MacTeX for MacOSX
\end{flushleft}


\begin{flushleft}
users). Includes the TeXworks editor.
\end{flushleft}


\begin{flushleft}
Workbook Source Files
\end{flushleft}


\begin{flushleft}
http://edin.ac/17EQPM1
\end{flushleft}


\begin{flushleft}
Download the .tex file and other files needed to compile this workbook.
\end{flushleft}





31





\newpage



\end{document}
